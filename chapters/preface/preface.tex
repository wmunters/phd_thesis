\chapter*{Preface}                                  \label{ch:preface}

Reeds toen in de middelbare school duidelijk werd dat ik aanleg had voor wetenschap, werd het me duidelijk: ik wil doctoreren, de grenzen van de kennis verleggen, de wereld voorgoed veranderen. Waarmee ik dit zou doen, of hoe ik dit zou aanpakken was me echter nog een raadsel. Doorheen mijn studies werd mijn na\"iviteit gelukkig wat getemperd, maar mijn focus werd verscherpt. De schoonheid van turbulente stromingen fascineerde me, en de keuze om onderzoek te doen naar de windstroming doorheen windturbineparken was voor mij snel gemaakt. Misschien was het me allemaal wat vroeger te binnen geschoten als ik wat aandachtiger geluisterd zou hebben wanneer mijn papa voor de zoveelste keer `Blowing in the wind' van Bob Dylan in de auto opzette. Papa, heb je al ooit ongelijk gehad? Hier ben ik dan, aan het eind van mijn doctoraat, en kijk ik terug op de vier meest verrijkende jaren van mijn leven tot nog toe. Uiteraard was dit alles niet mogelijk geweest zonder de hulp van de mensen rondom mij. Daarom wil ik hen graag op de volgende manier bedanken.

First and foremost, my deepest gratitude goes out to my advisor Johan Meyers. Thank you for giving me the opportunity to start this PhD. 
Your eagerness, knowledge and experience have always been an incentive to dig deeper into the results and to try and think outside the box. I believe working with an advisor like you is truly a privilege for any young researcher. I'd also like to thank my jury for carefully reading and evaluating my dissertation, and for the valuable discussions about my research. I wish to acknowledge the European Research Council for funding my PhD research through the ActiveWindFarms project. I also want to express my gratitude to the Flemish Supercomputer Center and the local KU Leuven HPC support team. Special thanks go out to Martijn Oldenhof, who was a great help in improving the parallel scalability of the LES solver used in this work.

%JHU
In the first year of my PhD, I had the opportunity of visiting Johns Hopkins University. I'd like to thank Charles Meneveau for hosting me and taking the time out of his busy schedule for weekly meetings. I also thank the other members of the Turbulence research group. A special word goes out to Adrien, who went out of his way to make me feel welcome from day one. My time in Baltimore has been an incredibly enriching chapter of my life. The experiences I've had and the people I've met will always have a warm place in my heart. 

%COLLEGA TME, happy hour, Vahid, ...
Also here in Leuven I was surrounded by a whole bunch of great people, many of which I truly consider to be friends. Juliaan, thank you for all your help involving the practicalities of moving to the US, including our weekly adventures to the Giant in Waverly. Dries, thanks for all the coffee breaks in which I could clear my clouded mind. Thanks for the discussions about work, research, or life in general. Thanos, thank you for the nice times we shared in our office, and for those outside in the real world. Vahid, who taught me valuable lessons in how to put life into perspective. Jay, Asim and Cornelia, for introducing me into the world of CFD and optimization. Jorge, No\'e, Geert, Kenneth, Maarten, Andreas, Iago, Pieter, Javier, Nicholas, Filip, Kris, Annelies, Sander, Roxane, Thomas, Dieter and all the others with whom I've shared serious or not-so-serious conversations over a beer. Tim and Gunther, for making my current office feel like Limburg everyday. A big thank you also goes out to all the other colleagues of the TFSO research group, the TME division and the entire department. Whether it were a cup of coffee, a game of football during lunch, a run after work or a chat at the happy hour, each of you contributed in his/her own way to making the time I spent doing my PhD unforgettable. I also want to acknowledge the administrative staff at the department for always being nice and helpful throughout these four years. 

%DE VRIENDEN
Uiteraard bedank ik ook mijn vrienden. Het leven is een kronkelend pad: bedankt om mij steeds de weg ernaar terug te helpen vinden, en om op tijd en stond samen met mij ervan af te gaan. Bedankt voor het delen van een luisterend oor, een knuffel, iedere lach maar ook elke traan. Bedankt voor de boysweekends in Oostende, de modderige festivalcampings, en de talloze optredens. Een volledige opsomming van mensen die me nauw aan het hart liggen zou van dit voorwoord een volwaardig hoofdstuk maken, maar ik wil toch graag Dimi, Nina, Tess, Sam, Sam, Ilse en Lionel bij naam noemen.

Mijn laatste woorden van dank wil ik aan mijn familie schenken. Mama, Papa, bedankt voor alle kansen die jullie me geschonken hebben. Zonder jullie steun en liefde zou ik hier nooit gestaan hebben. Marthe, mijn zus maar vooral mijn beste vriendin: bedankt om hard te zijn wanneer ik met mijn voeten op de grond gezet moet worden, maar ook om naar me te luisteren als je merkt dat het nodig is. Ik voel me werkelijk bevoorrecht om in zulk een warm gezin te mogen zijn opgegroeid. Dank jullie allemaal.

Wim Munters \hfill Leuven, December 2017

%%%%%%%%%%%%%%%%%%%%%%%%%%%%%%%%%%%%%%%%%%%%%%%%%%
% Keep the following \cleardoublepage at the end of this file, 
% otherwise \includeonly includes empty pages.
\cleardoublepage

% vim: tw=70 nocindent expandtab foldmethod=marker foldmarker={{{}{,}{}}}
