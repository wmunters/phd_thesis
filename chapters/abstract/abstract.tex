\chapter{Abstract}                                 \label{ch:abstract}

In wind farms, velocity deficits in the wakes originating from upstream turbines result in a significant decrease in power extraction in downstream turbine rows. The current control paradigm in industry optimizes power extraction at the turbine level, and does not account for these interactions, leading to sub-optimal wind-farm efficiencies. The current dissertation investigates the use of a coordinated control approach that maximizes power extraction at the wind-farm level instead. To this end, optimal control techniques are combined with a flow model based on large-eddy simulations in which turbines are modeled using an actuator disk approach. Furthermore, sensitivities of wind-farm power extraction to control parameters are evaluated in a tractable manner using an adjoint approach. In this way, individual wind turbines can be employed as dynamic flow actuators to both influence and harness specific flow features of the turbulent atmospheric boundary layer, with the aim of increasing the overall wind-farm power extraction. Although this approach is computationally infeasible for real-time wind-farm control, it allows to benchmark the potential of coordinated control. Furthermore, an understanding of the flow physics in optimized wind farms can lead to derived practical control strategies.

In a first step, the in-house solver SP--Wind is accelerated in order to obtain well-converged optimization results in reduced computational time. To this end, a pencil decomposition parallelization is implemented in order to efficiently run the solver on the order of a thousand cores, reducing the walltime for a large-eddy simulation with an order of magnitude. Furthermore, the classical conjugate-gradient optimization algorithm is upgraded to a quasi-Newton method, reducing the amount of functional and gradient evaluations required for a similar decrease in the cost function by a factor four. 

Next, the issue of generating unsteady and coherent turbulent inflow conditions for large-eddy simulations is addressed. The quality of precursor methods with periodic boundary conditions is illustrated based on a comparison with synthetic inflow turbulence methods. Furthermore, two limitations of these precursor methods are addressed. Firstly, the difficulties involved with variable wind directions are resolved through the definition of a novel generalized time-varying mean-flow precursor method. Secondly, the spurious spanwise inhomogeneities caused by periodic boundary conditions are eliminated through a new shifted periodic boundary condition for precursor simulations. 

Thereafter, a first set of optimizations investigates optimal induction control of a large wind farm with twelve rows of turbines in the turbulent atmospheric boundary layer. The induced slowdown of the incoming wind caused by each turbine is controlled by dynamically varying its thrust coefficient. For selected coordinated control cases, power gains in the order of 15 to 20\% are achieved with respect to a greedy reference case that maximizes power at the turbine level. Further analysis of flow fields and control dynamics results in the identification of a new dynamic induction control strategy which increases power in the entrance region of the wind farm through quasi-periodic vortex shedding. 

Finally, a second set of optimizations also includes the possibility of yawing turbines with respect to the incoming flow, e.g. to redirect wakes away from downstream turbines. Simulations are performed for a smaller wind farm with four rows, and both uniform and turbulent inflow conditions are investigated. Optimizations results show that, for the given wind farm, yaw control is more effective than induction control. Analysis of optimal yawing characteristics results in the definition of two derived control strategies, i.e. a steady yaw strategy redirecting wakes away from downstream turbines, and a dynamic strategy enhancing wake meandering in low-turbulence flows. Furthermore, the benefit of dynamic yaw control in fully-developed turbulence is shown. Finally, the potential of combining dynamic yaw control with dynamic induction control is quantified for the first time, leading to the largest power gains of all control cases, with values of 110\% and 34\% for uniform and turbulent inflow respectively. 



%%%%%%%%%%%%%%%%%%%%%%%%%%%%%%%%%%%%%%%%%%%%%%%%%%
% Keep the following \cleardoublepage at the end of this file, 
% otherwise \includeonly includes empty pages.
\cleardoublepage

% vim: tw=70 nocindent expandtab foldmethod=marker foldmarker={{{}{,}{}}}
