\chapter{Conclusion}\label{ch:conclusion}

The current dissertation investigated dynamic optimal control of wind farms for increased power extraction using large-eddy simulations combined with adjoint gradient evaluations, as an alternative to the currently prevailing greedy control paradigm. To this end, wind turbines were dynamically controlled with aim of increasing wind-farm power extraction through benign influence on the boundary-layer flow. The SP--Wind framework was upgraded to allow the simulation and control of more cases with better convergence. In addition to new dynamic induction control studies in line with prior work, also dynamic yaw control was investigated. Analysis of optimized induction and yaw characteristics led to the identification of flow mechanisms responsible for increasing power extraction, and resulted in simplified control approaches mimicking the behavior of the full optimal control studies.

The current chapter closes the dissertation by providing a summary, presenting general conclusions, and discussing the future outlook. First, Section~\ref{sec:conc_summ} summarizes the current work. Next, Section~\ref{sec:conc_key} presents the key take-away messages. Finally, Section~\ref{sec:conc_outlook} provides an outlook and recommendations for further research. 

\section{Summary}\label{sec:conc_summ}
Continuing on previous work, the SP--Wind framework for large-eddy simulation and optimization of wind farms was upgraded: the parallelization was converted to a pencil decomposition strategy improving parallel scalability, and the optimizer was equipped with an updated quasi-Newton optimization algorithm for better convergence. Furthermore, a fringe region technique was implemented to allow the simulation of non-cyclic spatially developing wind-farm flows, and yaw capabilities were implemented in the optimization framework.  

The issue of the generation of inflow conditions for turbulence-resolving flow simulations was addressed Based on numerical testing and a review of the literature, the importance of high quality inflow turbulence was reaffirmed, and the use of precursor inflow methods for wind-farm LES was justified. A generalization of concurrent precursor methods to time-varying mean-flow directions was presented and illustrated based on a LES study of the Horns Rev wind farm. Furthermore, a new shifted periodic boundary condition for precursor simulations was presented, eliminating spurious spanwise inhomogeneities polluting long-time averages of physically spanwise homogeneous flows. 

A first optimization study consisted of the optimal dynamic induction control of an aligned wind farm with $12 \times 6$ turbines. 
A suite of control cases was defined, based on differing wind turbine response times and whether or not overinduction was allowed. Except for the most restrictive case, limited to underinductive slow-response control, all control cases resulted in an increase in wind-farm energy extraction ranging from 8$\%$ to 21$\%$. It was shown that, also for cases with smooth turbine dynamics and reduced thrust loading variability, significant power increases could be achieved. A comparison of optimized flow-field statistics with a greedy reference wind farm revealed an increase in mean axial velocity at the wind turbines, improved turbulent momentum transport into the core of the wake, and higher overall turbulence levels throughout the wind farm. Furthermore, increased transversal mean-flow transport was found in the entrance region of the wind farm, and it was found that differences in flow-field statistics were most salient and coherent in the first few rows of the optimally controlled wind farms.

Further analysis was performed on the overinductive fast-response dynamic induction case C3t5, as it was observed to exhibit similar flow statistics as the maximum-yield instantaneous-response case C3t0, but with smoother thrust force dynamics. Analysis of the thrust coefficients and numerical experiments illustrated a distinction between first-row turbines, last-row turbines, and intermediate turbines. It was shown that optimized thrust coefficients are largely tuned to local flow conditions, and that only the first row contains thrust characteristics that in part do not depend on specific flow conditions.  Furthermore, observations suggested that the optimization works in a unidirectional way: upstream turbines influence the flow field resulting in favorable conditions for their downstream neighbors, but information on the possibility of active response and cooperation in the latter has no influence on upstream control actions. A qualitative analysis of instantaneous flow fields illustrated quasi-periodic shedding of vortex rings from first-row turbines in the controlled case, partially responsible for the power gain in downstream rows. This mechanism was mimicked through the use of a simple sinusoidal thrust control approach in a reduced-size wind farm, which led to a robust increase in second-row power extraction for periodic variations with a Strouhal number of 0.25. Furthermore, this mechanism could be linked to the first-row thrust characteristics found through the numerical experiments. In contrast to earlier work on steady axial induction control, for which power gains have often been marginal or non-existent, the current work hence results in a robust axial induction control strategy with significant power gains, irrespective of specific local flow conditions. This marks an important step towards practical dynamic axial induction control for increased power extraction in small sets of aligned wind turbines. However, the sinusoidal approach was shown not to work in downstream rows, and a full-scale wind-farm test further showed that active downstream turbines are required to increase total power extraction in extended wind farms.  

A second optimization study considered both dynamic induction control and dynamic yaw control for an aligned wind farm with $4 \times 4$ turbines.
Again, a suite of control cases was defined: an axial induction control cases with underinductive and overinductive behavior, a set of yaw control cases with standard and high yaw rates, and a set of cases combining induction control with yaw control. Both uniform and turbulent inflow conditions were considered. For the given wind farm it was found that, in both inflow regimes, yaw control was more effective than induction control. Furthermore, it was shown that combining overinduction control with yaw control further increased potential for power extraction, resulting in a relative increase of 110\% and 34\% for uniform and turbulent inflow respectively. Two distinct regimes in the yaw control cases were identified, i.e. a steady yaw wake redirection regime, and a dynamic yawing regime reinforcing wake meandering. Although the former regime was found to be robust to the turbulence levels in the inflow, the latter only achieved an increase in power extraction for uniform conditions. For uniform inflow conditions, it was found that the possibility of dynamic yawing adds little benefit over steady wake redirection through static yaw control. In contrast, for turbulent inflow conditions, dynamic yawing allows turbines to continuously adapt to unsteadiness in local flow conditions, resulting in a further increase in power extraction over static yaw control. 

In summary, the work presented in this thesis further developed and explored the use of optimal wind-farm control for increased power extraction using large-eddy simulations and adjoint-based gradient evaluation. New control cases and improved convergence of the optimizations led to increased power gains compared to previous work and facilitated the identification of flow mechanisms responsible for parts of these gains. 

%\section{Conclusions}\label{sec:conc_conc}
%In the current section, general conclusions are formulated based on observations spanning multiple chapters of the current thesis.
%
%First, the obtained results justify the application of computationally demanding but accurate simulation methods in deriving control strategies for wind farms. More specifically, large-eddy simulations are particularly well suited to capture complex dynamics such as the shedding of vortex rings observed by the dynamic induction control simulations, or the curling of wakes behind turbines in misalignment exploited by the optimal yaw control simulations. Moreover, the yaw control study also illustrated that the physics of optimally controlled wind-farm flows is highly dependent on the inflow turbulence, further justifying the application and development of accurate precursor inflow methods. 
%
%Second, although the current optimal control methodology is not applicable to wind farms in practice, both due to computational constraints and the lack of full state information in reality, analysis of the optimal control simulations has led to the identification of flow mechanisms in the wind-farm entrance region that could potentially be harnessed in practice, i.e. vortex ring shedding, curled wake redirection, and reinforced wake meandering. Further development of control strategies based on these phenomena may lead to wind-farm controllers that could be retro-fitted in currently operational wind farms. 
%
%Third, even though literature focuses mostly on wind-farm control through yaw or induction separately, the current results illustrate that the potential of combining yaw and induction for increased wind-farm power extraction. Although the mechanisms observed in this thesis for both control strategies are different, i.e. induction control acts to improve wake mixing whereas yaw control is mainly used for wake redirection, these mechanisms can be exploited together to further increase wind-farm power.

\section{Key take-away messages}\label{sec:conc_key}
The current section lists the key take-away messages of the current dissertation. First, the current work has shown the importance of precursor inflow methods for wind-farm large-eddy simulations in neutral boundary layers. Moreover, the generalization of precursor methods to varying inflow directions, and the newly developed shifted periodic boundary conditions are a general contribution to the challenge of generating turbulent inflow conditions. More specifically, they are not limited to wind-energy applications, as they can be applied to any type of turbulence-resolving simulation of wall-bounded flows.

Second, the analysis of the optimal induction control simulations featured in the current work has led to the identification of a new axial induction control strategy which increases wind-farm power extraction by quasi-periodic vortex ring shedding. To the best knowledge of the author, this is the first time that optimal control simulations of wind farms have succesfully been used to come up with a simple and robust control strategy that could be used in practice.

Third, the optimal dynamic yaw control simulations have illustrated a new dynamic yawing control approach that increases wind-farm power through reinforced wake meandering. Although it was shown that wake meandering is only reinforced in low ambient turbulence, the current observations could spark new studies that exploit these effects in more realistic flow conditions as well. Also in flows with higher ambient turbulence, dynamic yaw control proved useful in addition to steady wake-redirecting yaw. In this way, turbines can react to unsteady local flow conditions resulting in an increased power extraction compared to steady yaw control. 

Finally, it was shown that, through the combination of dynamic yaw control with dynamic axial induction control, power extraction can be increased significantly compared to the case in which only one of these controls was used. The current work marks the first study in which the potential of combining these dynamic control strategies has been quantified in a high fidelity modeling environment. 

\section{Outlook and recommendations for further research}\label{sec:conc_outlook}
The current work applied optimal control techniques to wind-farm boundary layers with the aim of increasing total power extraction. Significant potential power gains were quantified, and simplified control strategies that do not require computationally expensive optimizations were identified. Nevertheless, some simplifications were made, and several challenges remain before the results of this thesis could be translated to real wind farms.  The current section discusses limitations of the current approach and provides associated recommendations for further research. 

It is important to note that, throughout the current thesis, a neutral rough-wall half-channel flow at very high Reynolds number was used as a surrogate for the atmospheric boundary layer. The effects of Coriolis forces and thermal stability are hence omitted. It is well known in literature that these effects can have strong implications on wind-farm flows. For instance, thermal stability of the ABL has an important effect on wake recovery and power extraction.  Especially wind farms in stable ABLs are of interest, as they are characterized by very long wakes with large velocity deficits and suppressed wake meandering \citep{larsen2009dependence,machefaux2016experimental}, resulting in larger power losses compared to neutral conditions \citep{barthelmie2010evaluation, dorenkamper}, and power transients during the transition from a neutral to stable ABL in the evening \citep{allaerts2017gravity}. The application of optimal wind-farm in stable ABLs hence holds significant potential for increasing farm power. Moreover, power gains could possibly be achieved through different physical mechanisms than those identified in the current thesis for the neutral case. The main step towards running optimal wind-farm control simulations in actual ABLs would be the derivation and implementation of the adjoint transport equation for potential temperature. 

Turbines are represented using a relatively simple non-rotating actuator disk model parametrized by a single thrust coefficient. Although it has been shown that ADMs provide accurate far-wake flow characteristics, the absence of rotation and a discrete blade representation have been observed to result in discrepancies with measurements in near-wake behavior \citep{wu2011large}. These discrepancies can be avoided by using more advanced wind-turbine models, such as actuator sector models or actuator line models. In addition to more accurate near-wake physics, these models also allow control of the generator torque and blade pitch angles as in real wind turbines, instead of directly controlling the resulting thrust coefficient as in the current thesis. This is subject of ongoing work, although the additional spatio-temporal resolution requirements associated with such models limit current studies to problems of reduced size, e.g. with two aligned wind turbines only. Next to that, the credibility of the new simplified control strategies for the first-row ADM in this thesis, i.e. the periodic vortex ring shedding through sinusoidal induction control and the reinforced wake meandering through alternating yaw control in low-turbulence flows, could be further confirmed using wind-tunnel testing. 

The optimal control simulations in this work were designed to optimize wind-farm power extraction at any cost. In reality, wind-farm control however can involve multiple additional conflicting objectives such as tracking a reference mean power signal, reducing power variability, or minimizing turbine loads. With respect to the latter objective, it is important to note that the controls developed in this thesis might result in unacceptable additional loads. Therefore, an important future step is to assess turbine loading in the proposed control strategies. Recently, the turbine models in SP--Wind framework have been expanded with aero-elastic features \citep{vitsas2016multiscale}. An interesting area of future research would be to perform multi-disciplinary multi-objective optimizations, resulting in Pareto-optimal control strategies that take into account aerodynamics, structural mechanics, and electricity grid constraints.

The analysis of optimized thrust coefficients and yaw angles has led to the aforementioned simplified control strategies for first-row turbines. However, no such strategies were found for the intermediate wind turbines which operate in a more complex environment. It is expected that also the mechanisms behind the optimized controls are more complex, and more advanced analysis tools are required in order to identify them. An interesting route here is to apply proper orthogonal decomposition of flow fields and control signals, provided that a longer total time is simulated for adequate statistical convergence of the decomposition. The dominant modes in the control signals should then be correlated to specific flow events. Also, the use of longer time horizons in the receding-horizon approach, e.g. corresponding to a flow-through time of the full wind farm instead of only four turbine rows in this thesis, could prove instrumental to obtaining better control signals for further analysis. Due to the inherent ill-posedness in adjoint sensitivity problems of chaotic turbulent flows, this would require an alternative gradient evaluation approach, such as least squares shadowing technique, and extensions thereof \citep{wang2014least, chater2016simplified, blonigan2017multigrid}.

It was mentioned throughout this thesis that the current approach is infeasible for practical wind-farm control, and that the aim of this approach is to benchmark the potential for power increase and result in a physical understanding of controlled wind farms. However, given increasingly powerful supercomputers, the current LES evaluations achieve parity between simulation `wind-farm' time and computational wall time. Provided that computational power continues its growth, it is expected that LES on relatively coarse grids can be performed in a timely manner for control purposes as well in the near future. Upon the implementation of a state estimation algorithm and a feedback loop to correct for model mismatch between the coarse LES and the wind-farm flow (be it in the form of a `virtual' high-resolution wind-farm LES or an actual wind farm), LES-based wind-farm control could potentially be implemented in practice. This marks a final interesting area for further research.


%%%%%%%%%%%%%%%%%%%%%%%%%%%%%%%%%%%%%%%%%%%%%%%%%%
% Keep the following \cleardoublepage at the end of this file, 
% otherwise \includeonly includes empty pages.
\cleardoublepage

% vim: tw=70 nocindent expandtab foldmethod=marker foldmarker={{{}{,}{}}}
