\chapter*{Beknopte samenvatting}

In windturbineparken resulteert de verlaagde windsnelheid in het zog van stroomopwaartse turbines in een significant verlies van energiewinning door stroomafwaartse turbines. De huidige industri\"ele controlemethodologie optimaliseert energiewinning op het niveau van individuele turbines en houdt geen rekening met deze interacties, hetgeen leidt tot sub-optimale effici\"entie voor windturbineparken. In de plaats hiervan, bestudeert dit proefschrift de toepassing van een geco\"ordineerde controlemethodologie die de totale energiewinning op het niveau van het windturbinepark maximaliseert. Hiertoe worden optimale controletechnieken gecombineerd met een stromingsmodel gebaseerd op large-eddy simulations (\emph{NL: grote-wervel simulaties}) waarin de turbines gemodelleerd worden door een actuator disk model (\emph{NL: actuator schijf model}). Daarenboven wordt de invloed van controleparameters op de totale energiewinning door het windturbinepark berekend aan de hand van een geadjungeerde gradi\"ent die de rekenkost handelbaar maakt. Op deze manier kunnen individu\"ele windturbines gebruikt worden als actieve stromingsactuatoren die zowel een invloed uitoefenen op als gebruik maken van specifieke stromingsfenomenen in de turbulente atmosferische grenslaag, teneinde de totale energiewinning door het windturbinepark te verhogen. Hoewel de hoge rekenkost van deze aanpak vandaag een rechtstreekse toepassing van deze methodologie in de praktijk belemmert, laat hij toe de mogelijkheden van geco\"ordineerde controle te kwantificeren, en kan een beter begrip van stromingsfysica in geoptimaliseerde parken leiden tot afgeleide praktische controlestrategie\"en. 

Eerst wordt het genereren van transi\"ente en coherente turbulente instroom voor large-eddy simulations bestudeerd. De kwaliteit van zogenaamde precursormethodes met periodieke randvoorwaarden wordt geillustreerd aan de hand van een vergelijking met artifici\"ele turbulente instroom. Daarenboven worden twee beperkingen van klassieke precursormethodes verholpen. Ten eerste wordt het inherente probleem van precursormethodes met variable windrichtingen opgelost door de ontwikkeling van een veralgemeende precursormethode met tijdsvariabele instroomrichting. Ten tweede worden onfysische transversale inhomogeniteiten ten gevolge van de periodieke randvoorwarden ge\"elimineerd door middel van een nieuwe verschoven periodieke randvoorwaarde voor precursorsimulaties. 

Vervolgens bestudeert een eerste reeks optimalisaties de optimale inductiecontrole van een groot windturbinepark met twaalf windturbinerijen in de turbulente atmosferische grenslaag. De ge\"induceerde vertraging van de aanstromende wind veroorzaakt door elke windturbine wordt dynamisch gecontroleerd door middel van haar stootco\"efficient. In bepaalde gevallen leidt geco\"ordineerde controle tot een toename in energiewinning in de grootteorde van 15\% tot 20\% behaald, in vergelijking met een referentiecontrole die energiewinning maximaliseert op turbineniveau. Verdere analyse van stromingsvelden en controledynamica leidt tot het identificeren van een nieuwe dynamische inductiecontrolestrategie waarin energiewinning in de inlaatzone van een windturbinepark verhoogt kan worden door middel van quasi-periodische afscheiding van ringvormige draaikolken. 

Hierna beschouwt een tweede reeks optimalisaties tevens de mogelijkheid om de turbinerotor te verdraaien van de aanstromende windrichting (zijnde een gierbeweging), bijvoorbeeld om haar zog weg te leiden van stroomafwaartse turbines. Simulaties worden uitgevoerd voor een kleiner windturbinepark met vier rijen, en zowel uniforme als turbulente instroom wordt onderzocht. De optimalisaties tonen aan dat, voor het gegeven turbinepark, giercontrole effectiver blijkt dan inductiecontrole. Analyse van optimale gierbewegingen resulteert in de ontwikkeling van twee afgeleide controlestrategie\"en, met enerzijds een constante hoekverdraaiing die het zog van stroomafwaartse turbines wegleidt, en anderzijds een dynamische gierbeweging die de meanderbeweging van het zog versterkt in stromingen met beperkte turbulentie. Daarenboven wordt ook het voordeel van dynamische giercontrole in volledig ontwikkelde turbulentie aangetoond. Ten slotte wordt het potentieel van gecombineerde dynamische inductie- en giercontrole voor de eerste maal gekwantificeerd, resulterend in de hoogste energiewinning van alle controlestudies, met toenames van respectievelijk 110\% en 34\% voor uniforme en turbulente inlaatstroming. 
%%%%%%%%%%%%%%%%%%%%%%%%%%%%%%%%%%%%%%%%%%%%%%%%%%
% Keep the following \cleardoublepage at the end of this file, 
% otherwise \includeonly includes empty pages.
\cleardoublepage

% vim: tw=70 nocindent expandtab foldmethod=marker foldmarker={{{}{,}{}}}
