\chapter{Variance estimation of energy gains}\label{ch:app_variance}

In this appendix, the estimation for the variance of the energy gains in the optimal induction control simulations in Chapter \ref{ch:opt_induction} is presented. 
This estimation of the variance of the ratio between optimized and reference energy extraction $E_O/E_R$ is based on $N_w$ optimization windows, where $N_w = 14$, including all windows except the first, which exhibits transient behavior for the optimized cases. Because the integral time scale of power extraction $\Lambda \approx 50$ s is significantly smaller than the window time horizon $T_A = 120$ s, time-integrated quantities over different windows can be assumed to be statistically independent. Firstly, the total time-averaged power is defined as 
\begin{equation}
\overline{P} = \frac{1}{T_\text{tot}} \int_{0}^{T_\text{tot}} P(t)~\text{d}t = \frac{E}{T_\text{tot}},
\end{equation}
hence $E_O/E_R = \overline{P}_O / \overline{P}_R$. Furthermore, the window-averaged power extraction for each optimization window $i$ is 
\begin{equation}
\overline{P}_{i}^{T_A} = \frac{1}{T_A} \int_{i\ T_A}^{(i+1)T_A} P(t)~\text{d}t \qquad \text{for}~i = 1 \dots N_w.
\end{equation}
\noindent Using a Taylor series approximation for the ratio $\overline{P}_{\text{O}} / \overline{P}_R$, the variance thereof can be expressed in terms of statistical quantities of its numerator and denominator (see, e.g., \citealp{kendalladvanced}) as 
\begin{equation}\label{eq:formula}
\text{var} (\overline{P}_O / \overline{P}_R) \approx \frac{\text{E}(\overline{P}_O)^2}{\text{E}(\overline{P}_R)^2} \bigg[ \frac{\text{var}(\overline{P}_O)}{\text{E}(\overline{P}_O)^2} - 2\frac{\text{cov}(\overline{P}_O, \overline{P}_R)}{\text{E}(\overline{P}_{R})\text{E}(\overline{P}_O)} + \frac{\text{var} (\overline{P}_R)}{\text{E}(\overline{P}_R)^2}  \bigg],
\end{equation}
\noindent where var, cov and E denote the variance, covariance and expected value respectively. Since $\overline{P} = (\sum_{i=1}^{N_w}\overline{P}_i^{T_A})/N_w$, the variances in Equation \eqref{eq:formula} can be written as $\text{var}(\overline{P}) = \text{var}(\overline{P}^{T_A})/N_w$. Furthermore, combining properties of the covariance operator with the fact that mean powers in different optimization windows can be assumed to be statistically independent, the covariance in Equation \eqref{eq:formula} can be rewritten as 
\begin{align}
	\text{cov}(\overline{P}_O, \overline{P}_R) &= \text{cov} \Bigg[\bigg(\sum_{i=1}^{N_w}\overline{P}_{O,i}^{T_A}\bigg)/N_w, \bigg(\sum_{j=1}^{N_w}\overline{P}_{R,j}^{T_A}\bigg)/N_w \Bigg] \\
	&= \frac{1}{N_w^2} \sum_{i=1}^{N_w} \sum_{j=1}^{N_w} \text{cov}\bigg(\overline{P}_{O,i}^{T_A}, \overline{P}_{R,j}^{T_A} \bigg)\\
	&= \frac{1}{N_w^2} \sum_{i=1}^{N_w} \text{cov}\bigg(\overline{P}_{O,i}^{T_A}, \overline{P}_{R,i}^{T_A} \bigg)\\
	&= \frac{1}{N_w} \text{cov} \bigg( \overline{P}_O^{T_A}, \overline{P}_{R}^{T_A} \bigg).
\end{align}
\noindent Finally, since $\text{E}(\overline{P}) = \overline{P}$, Equation \eqref{eq:formula} can be readily evaluated in terms of quantities that can either be directly obtained from total time-averaged powers $\overline{P}_O$ and $\overline{P_R}$, or calculated from the distribution of window-averaged powers $\overline{P}^{T_A}_O$ and $\overline{P}^{T_A}_{R}$ as
\begin{equation}\label{eq:formula2}
\text{var} (\overline{P}_O / \overline{P}_R) \approx \frac{1}{N_w} \frac{\overline{P}_O^2}{\overline{P}_R^2} \bigg[ \frac{\text{var}(\overline{P}_O^{T_A})}{\overline{P}_O^2} - 2\frac{\text{cov}(\overline{P}_O^{T_A}, \overline{P}_R^{T_A})}{\overline{P}_R \overline{P}_O} + \frac{\text{var} (\overline{P}_R^{T_A})}{\overline{P}_R^2}  \bigg],
\end{equation}
leading to a standard deviation $\sigma = (\text{var} (\overline{P}_O / \overline{P}_R))^{1/2}$. Note that, even though $\overline{P}_O$ and $\overline{P}_R$ are normally distributed, the ratio between them is not. Therefore, the interval of $\text{E}(\overline{P}_O / \overline{P}_R) \pm 2 \sigma$, as shown in the error bars in Figure \ref{fig:bar_and_row}, does not correspond to a 95\% confidence interval as would be the case for a normal distribution of $\overline{P}_O / \overline{P}_R$. Nonetheless, by definition of the standard deviation, the $\text{E}(\overline{P}_O / \overline{P}_R) \pm 2 \sigma$ interval provides a good uncertainty estimate for the gains reported in Figure \ref{fig:bar_and_row}.

\cleardoublepage